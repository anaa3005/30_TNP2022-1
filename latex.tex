\documentclass[a4paper,12pt]{article}

\usepackage{color}
\usepackage{url}
\usepackage[T2A]{fontenc} 
\usepackage[utf8]{inputenc} 
\usepackage{graphicx}

\usepackage[english,serbian]{babel}

\usepackage[unicode]{hyperref}
\hypersetup{colorlinks,citecolor=green,filecolor=green,linkcolor=blue,urlcolor=blue}

\newtheorem{primer}{Primer}[section]

\begin{document}

\title{Pametni gradovi\\ \small{Seminarski rad u okviru kursa\\Tehničko i naučno pisanje\\ Matematički fakultet}}

\author{Petar Deljanin\\ deljanin.petar2004@gmail.com\\Ana Crnomarković\\ anacrnomarkovic50@gmail.com\\ Aleksandar Živanović\\ aleksandar.zivanovic39@gmail.com\\ Nina Stamatović\\ nina.stamatovic03@gmail.com}
\date{11.~novembar 2022.}
\maketitle

\abstract{
U ovom tekstu je ukratko prikazana osnovna forma seminarskog rada. Obratite pažnju da je pored ove .pdf datoteke, u prilogu i odgovarajuća .tex datoteka, kao i .bib datoteka korišćena za generisanje literature. Na prvoj strani seminarskog rada su naslov, apstrakt i sadržaj, i to sve mora da stane na prvu stranu! Kako bi Vaš seminarski zadovoljio standarde i očekivanja, koristite uputstva i materijale sa predavanja na temu pisanja seminarskih radova. Ovo je samo šablon koji se odnosi na fizički izgled seminarskog rada (šablon koji \emph{morate} da ispoštujete!) kao i par tehničkih pomoćnih uputstava. 

\tableofcontents

\newpage

\section{Uvod}
\label{sec:uvod}
Gradovi su glavni polovi ljudske i ekonomske aktivnosti. Oni imaju potencijal da stvore sinergiju omogućavajući velike razvojne mogućnosti njihovim stanovnicima. Međutim, oni takođe stvaraju širok spektar problema koji mogu biti teško rešiti kako rastu u veličini i složenosti. Gradovi su i mesta gde su nejednakosti jače i, ako se njima ne upravlja, njihovi negativni efekti mogu prevazići pozitivne. \\

Urbana područja treba da upravljaju svojim razvojem, podržavajući ekonomsku konkurentnost, istovremeno jačajući socijalnu koheziju, održivost životne sredine i povećan kvalitet života svojih građana. \\

Sa razvojem novih tehnoloških inovacija – uglavnom IKT-a – koncept „pametnog grada“ se pojavljuje kao sredstvo za postizanje efikasnijih i održivijih gradova. \\

Od svoje koncepcije, koncept pametnog grada je evoluirao od izvođenja konkretnih projekata do implementacije globalnih strategija za rešavanje širih gradskih izazova. Stoga je neophodno dobiti sveobuhvatan pregled raspoloživih mogućnosti i povezati ih sa specifičnim izazovima grada.


\section{Koncept pametnih gradova}
Vaš seminarski rad mora da sadrži najmanje jednu sliku, najmanje jednu tabelu i najmanje tri reference u spisku literature. \textbf{Dužina seminarskog rada treba da bude:}
\begin{itemize}
\item Ukoliko tim ima dva člana, tada od 3 do 5 strana
\item Ukoliko tim ima tri člana, tada od 4 do 6 strana
\end{itemize} 

Iako postoji neka vrsta konsenzusa da oznaka Smart Citi predstavlja inovaciju u upravljanju gradom, njegovim uslugama i infrastrukturama, zajednička definicija pojma još nije data. Postoji širok spektar definicija šta bi pametni grad mogao biti. Međutim, dva trenda se mogu jasno razlikovati u vezi sa glavnim aspektima koje pametni gradovi moraju uzeti u obzir. \\

S jedne strane, postoji niz definicija koje stavljaju akcenat samo na jedan urbani aspekt (tehnološki, ekološki, itd.), ostavljajući po strani ostale okolnosti u gradu. Ova grupa monotopskih opisa je nesporazum da je krajnji cilj pametnog grada da obezbedi novi pristup urbanom upravljanju u kome se svi aspekti tretiraju uz međusobnu povezanost koja se dešava u stvarnom životu grada. Poboljšanje samo jednog dela urbanog ekosistema ne znači da se rešavaju problemi celine. \\

S druge strane, postoje neki autori koji naglašavaju kako je glavna razlika koncepta Smart Citi povezanost svih urbanih aspekata. Zamršeni problemi između urbanizacije su infrastrukturni, društveni i institucionalni u isto vreme i ovo preplitanje se ogleda u konceptu pametnog grada. Iz definicija se može primetiti da je infrastruktura centralni deo Pametnog grada i da je tehnologija onaj koji to omogućava, ali je kombinacija, povezanost i integracija svih sistema ono što postaje osnovno da bi grad bio zaista pametan. Iz ovih definicija može se zaključiti da koncept pametnog grada podrazumeva sveobuhvatan pristup upravljanju i razvoju grada. Ove definicije pokazuju ravnotežu tehnoloških, ekonomskih i društvenih faktora uključenih u urbani ekosistem. Definicije odražavaju holistički pristup urbanim problemima koristeći prednosti novih tehnologija tako da se urbani model i odnosi među zainteresovanim stranama mogu redefinisati.\\



\section{Izazovi pametnih gradova}	
\label{sec:termini_i_citiranje}

 Kako gradovi i dalje neumorno rastu, njihove izazove treba pažljivo razmotriti kako bi rast stanovništva, ekonomski razvoj i društveni napredak išli istim putem. Iako se većina globalnog BDP-a proizvodi u gradovima, sve što se dešava unutar ovih aglomeracija ne implicira pozitivne eksternalije. Gradovi su i mesta gde su nejednakosti jače i, ako se njima ne upravlja, negativni efekti mogu prevazići pozitivne. Model Smart Citi može dovesti do boljeg planiranja i upravljanja gradom, a samim tim i do postizanja održivog modela urbanog razvoja. \\

U ASCIMER-ovoj prvoj godini rada, izazovi su identifikovani i klasifikovani u različite dimenzije kako bi se olakšali naredni koraci projekta. Analizirajući urbanu sredinu, istraživački radovi se bave različitim brojem oblasti za uokvirivanje grada. U recenziranoj literaturi smo identifikovali da se svi oni mogu rasporediti u okviru šest glavnih gradskih dimenzija: upravljanje, ekonomija, mobilnost, životna sredina, ljudi i život (Giffinger, 2007). \\

Oni predstavljaju specifične aspekte grada na koje pametne inicijative utiču da bi se postigli očekivani ciljevi strategije pametnog grada (održivost, efikasnost i visok kvalitet života). Tehnologija se sama po sebi ne smatra akcionim poljem, već sredstvom koje poboljšava efikasnost projekata. \\

Unutar svake od dimenzija identifikovani su različiti gradski izazovi kako za gradove severnog Mediterana, tako i za gradove južnog i istočnog Mediterana. Gradovi koji se u ovom radu smatraju pripadajućim regionu Severnog Mediterana su oni koji se nalaze u zemljama Evropske unije. Zemlje u regionu južnog i istočnog Mediterana koje su razmatrane u studiji su: Maroko, Alžir, Tunis, Libija, Egipat, Jordan, Izrael, Liban, Sirija i Turska. \\

Ukupno je identifikovano dvadeset devet izazova za severne. Među njima, dvadeset se odnosi na samo jednu Dimenziju. I devet višedimenzionalnih izazova. Za južne gradove identifikovano je dvadeset izazova, od kojih se jedanaest odnosi na samo jednu dimenziju grada, dok ostalih devet odgovara dve ili više.\\


\begin{primer}
Problem zaustavljanja (eng.~{\em halting problem}) je neodlučiv \cite{haltingproblem}.
\end{primer}

\begin{primer}
Za prevođenje programa napisanih u programskom jeziku C može se koristiti GCC kompajler \cite{gcc}.
\end{primer}

\begin{primer}
 Da bi se ispitivala ispravost softvera, najpre je potrebno precizno definisati njegovo ponašanje \cite{laski2009software}. 
\end{primer}

Ukoliko za unos referenci koriste datoteku {\em seminarski.bib},  prevođenje u pdf format u Linux okruženju može se uraditi na sledeći način:

Prvo latexovanje je neophodno da bi se generisao {\em .aux} fajl. {\em bibtex} proizvodi odgovarajući {\em .bbl} fajl koji se koristi za generisanje literature. 
Potrebna su dva prolaza (dva puta pdflatex) da bi se reference ubacile u tekst (tj da ne bi ostali znakovi pitanja umesto referenci). Dodavanjem novih referenci potrebno je ponoviti ceo postupak.  


Broj naslova i podnaslova je proizvoljan. Neophodni su samo Uvod i Zaključak. Na poglavlja unutar teksta referisati se po potrebi. 
\begin{primer}
U odeljku \ref{sec:naslov1} precizirani su osnovni pojmovi, dok su zaključci dati u odeljku \ref{sec:zakljucak}.
\end{primer}




\section{Analiza projekata pametnih gradova}
\label{slike_i_tabele}

Različiti projekti pametnih gradova analizirani su na osnovu rezultata prethodne studije o konceptu pametnog grada i izazovima sa kojima se gradovi moraju suočiti. Analiza je podeljena u dve faze; prvo je razvijen konceptualni okvir koji će se koristiti kao orijentacija kroz mogućnosti razvoja Smart Citi projekata u različitim već objašnjenim dimenzijama. Drugo, detaljan opis odabrane grupe projekata i gradova koji precizira: kojoj vrsti akcije Smart Citi projekat pripada i koje su povezane gradske dimenzije koje on obuhvata; kakve gradske izazove pokušavaju da reše; i osnovne informacije o gradu u kojem je projekat sproveden. Osim toga, izrađeno je kratko objašnjenje samog projekta uključujući, kada je to moguće, stopu razvoja i obim projekta; kako se finansira; njegove ključne karakteristike inovacije i njegove glavne uticaje. \\

Evolucija koncepta Smart Citi vodi od konkretnih projekata do globalnih gradskih strategija kroz koje je moguće odgovoriti na izazove grada na različitim nivoima (nacionalnom, regionalnom, međunarodnom). Stoga je uočeno da je neophodno razviti strategiju u okviru grada za artikulisanje projekata u različitim dimenzijama kako bi se postigla holistička i sveobuhvatna vizija. Shodno tome, pored analize izolovanih akcija, identifikovane su i analizirane i neke izvanredne mediteranske strategije. Balans grada u 6 dimenzija je presudan za dobar učinak. Bez globalne strategije, grad je u opasnosti da izvede neke akupunkturne projekte koji dovode do toga da postane neuravnotežen, a samim tim i do drastičnog smanjenja uticaja ovih projekata. \\

Sve ove informacije su prikupljene u Vodiču za projekte koji tek treba da bude objavljen. Međutim, u ovom radu biće prikazani samo rezultati konceptualnog okvira i analiza nezavisnih projekata.\\



\begin{table}[h!]
\begin{center}
\caption{Razlčita poravnanja u okviru iste tabele ne treba koristiti jer su nepregledna.}
\begin{tabular}{|c|l|r|} \hline
centralno poravnanje& levo poravnanje& desno poravnanje\\ \hline
a &b&c\\ \hline
d &e&f\\ \hline
\end{tabular}
\label{tab:tabela1}
\end{center}
\end{table}

\\


\appendix
\section{Zaključak}
\label{sec:zakljucak}
Tokom ove prve godine, ASCIMER projekat je bio fokusiran na razvoj konceptualnog okvira za metodologiju procene za Smart Citi projekte u regionu Mediterana. Razumevanje i klasifikacija akcionih polja i postojećih projekata Pametnog grada bili su glavni rezultati projekta ove godine. \\

Kao kompleksan i višestruki koncept, nekoliko tipova projekata je definisano pod ovim kišobranom i stoga postaje neophodno odabrati glavne karakteristike koje projekat Smart citi mora imati. Projekti pametnih gradova moraju biti višedimenzionalni i integrisati različita polja delovanja grada, u interakciji sa ljudskim i društvenim kapitalom. Tehnološka rešenja se moraju shvatiti kao sredstvo za postizanje ciljeva pametnog grada i za suočavanje sa izazovima sa kojima se gradovi moraju suočiti. Glavni ciljevi Smart Citi projekata moraju biti rešavanje urbanih problema na efikasan način kako bi se poboljšala održivost grada i kvalitet života njegovih stanovnika. Sa stanovišta upravljanja, projekti moraju biti uokvireni u partnerstvu sa više zainteresovanih strana, na opštinskom nivou kako bi se obezbedila kompleksna i efikasna rešenja. \\

Glavni zahtev za projekte pametnih gradova mora biti rešavanje stvarnih izazova sa kojima će se gradovi suočiti u budućnosti. Ovo je prvi korak koji metodologija procene mora uzeti u obzir. Kada se analizira mediteranski region, ključno je razumeti različite izazove sa kojima se gradovi u evropskom kontekstu i u južnom i istočno-mediteranskom regionu moraju suočiti i na koji način su oni povezani. Izazovi sa kojima se evropski gradovi danas suočavaju mogu postati budući izazovi na jugu ako se njihovi sadašnji ne pozabave, uključujući ovu viziju. Projekti pametnih gradova moraju se pozabaviti problemima današnjih gradova, a istovremeno se osvrnuti na potencijalne probleme sa kojima će se gradovi suočiti u narednim decenijama. \\

Metodologije procene su neophodne za procenu stvarnog uticaja Smart projekata. Klasifikacija postojećih rešenja i projekata je glavni korak za postavljanje aspekata koje metodologija mora da proceni. Ovi aspekti moraju biti vezani za prethodno definisane izazove, shvatajući na koji način daju rešenje za probleme grada. Pružanje primera u svakoj od oblasti, u vezi sa ovim projektnim aktivnostima i izazovima, rezultira alatom za razvoj rešenja za probleme grada sa višedimenzionalnim i sveobuhvatnim pristupom. \\

Uzimajući u obzir nalaze o izazovima grada i analizu projekata ove prve godine, biće razvijena metodologija procene kako bi se procenili projekti pametnog grada u regionu južnog i istočnog Mediterana. Razvoj indikatora prilagođenih glavnim karakteristikama, projektima i izazovima gradova mediteranskog regiona i definisanje odnosa između njih za razvoj ispravne metodologije biće glavni cilj ASCIMER-a tokom naredne godine.\\



\end{document}

